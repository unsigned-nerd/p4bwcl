\documentclass[a4paper]{article}

% {{{ [begin] thai language support
% http://thaitug.daytag.org/wordpress/?p=1947
\usepackage{fontspec}
\usepackage{polyglossia}
\usepackage[Latin,Thai]{ucharclasses}
\defaultfontfeatures{Mapping=tex-text} 
\setdefaultlanguage{thai}
\setotherlanguage{english}
\newfontfamily{\thaifont}[Scale=MatchUppercase,Mapping=tex-text]{Norasi:script=thai}
%\newfontfamily{\thaifont}[Scale=MatchUppercase,Mapping=tex-text]{TlwgTypewriter:script=thai}
%\newfontfamily{\thaifont}[Scale=MatchUppercase,Mapping=tex-text]{TlwgTypewriter}
% if omitting script=thai, then we don't have to use ItalicFont,
% BoldFont, BoldItalicFont options
%\newfontfamily{\thaifont}[Scale=MatchUppercase,Mapping=tex-tex,ItalicFont={TlwgTypewriter-Oblique},BoldFont={TlwgTypewriter-Bold},BoldItalicFont={TlwgTypewriter-BoldOblique}]{TlwgTypewriter:script=thai}
%\newfontfamily{\thaifont}[Scale=MatchUppercase,Mapping=tex-text]{Purisa:script=thai}
\setTransitionTo{Thai}{\thaifont}
\setTransitionFrom{Thai}{\normalfont}
\XeTeXlinebreaklocale “th_TH” 
\XeTeXlinebreakskip = 0pt plus 1pt
\linespread{1.5}
% }}} [end] thai language support

\usepackage{hyperref}
\usepackage{fancybox}

\begin{document}

\title{Programming For Beginners With Common~Lisp}
\author{unsigned\_nerd}
\maketitle

\tableofcontents

\section{Prerequisite - การเริ่มต้น}

การเขียนโปรแกรม คือ การเขียนชุดคำสั่ง เพื่อสั่งให้คอมพิวเตอร์ทำงานตามที่เรากำหนด

เราเขียนชุดคำสั่งด้วย Programming Language

โดยทั่วไปหากเราพูดถึง Programming Language เราจะหมายถึงภาษาแบบ High-Level เช่น
Common Lisp, Python, Java, PHP, Perl, C, JavaScript เป็นต้น
แต่ไม่ได้หมายถึงภาษา Assembly

ภาษา Common Lisp เป็นภาษาที่ปัจจุบันไม่ได้รับความนิยมมากนัก

สามารถดูการจัดอันดับภาษาคอมฯที่ได้รับความนิยมล่าสุดได้ที่นี่: \href{http://pypl.github.io/PYPL.html}{http://pypl.github.io/PYPL.html}

อันดับหนึ่งคือภาษา Python

ส่วนภาษา Common Lisp ไม่ติดอันดับบน PYPL เลย

ทาง PYPL แนะนำว่าถ้าหากอยากทราบความนิยมของภาษาที่ไม่ติดอันดับให้ใช้ Google Trends ดู

เมื่อลองใช้ Google Trends เปรียบเทียบความนิยมระหว่าง 3 ภาษา ได้แก่ Python, PHP
และ Common Lisp เราจะพบว่ากระแสความนิยมในภาษา Common Lisp นั้นต่ำมาก
ดังจะเห็นได้จากภาพประกอบด้านล่าง


อย่างไรก็ตาม แม้ว่า Common Lisp จะไม่ได้รับความนิยมในปัจจุบัน
แต่ก็ไม่ได้หมายความว่ามันจะไม่เหมาะกับการใช้เป็นเครื่องมือในการเรียนรู้การเขียนโปรแกรม%
สำหรับมือใหม่

เมื่อท่านได้เรียนรู้การเขียนโปรแกรมด้วยภาษา Common Lisp แล้ว ท่านจะสามารถเรียนภาษา%
คอมฯอื่นๆเพิ่มเติมได้ง่ายขึ้นมาก

เป็นที่ยอมรับกันโดยทั่วไปว่าในชีวิตหนึ่งๆของคนเรา เราย่อมต้องเรียนภาษาคอมพิวเตอร์หลายภาษา
แล้วทำไมไม่ลองเริ่มด้วยภาษา Common Lisp กันเล่า!

\shadowbox{\begin{minipage}{0.9\linewidth}
{\Large รู้หรือไม่?}
ภาษา Logo (เจ้าเต่าน้อย Logo) ที่หลายๆท่านอาจได้เคยเรียนในสมัยเด็กๆเป็น
Dialect หนึ่งของภาษา Lisp นี่ก็ยิ่งชี้ให้เห็นว่า Common Lisp
เหมาะกับการเป็นภาษาเขียนโปรแกรมแรกๆของทุกท่านขนาดไหน
\end{minipage}}

\end{document}
